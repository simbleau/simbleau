\section{Justification of the Problem}

There are many benefits to vector graphics, but support for interactivity just isn't there. Computer graphic enthusiasts have been releasing experimental solutions for years in an attempt to render at interactive rates. However, there is a felt lack of of comparison connecting modern models. Our research would expel the mystery of these technologies.\\

\subsection{Pre-computation models}

One solution to optimize static 2D GPU path rendering is colloquially referred to as a "pre-compution" model, wherein, upfront computational expense is paid to cache a GPU-friendly representation. A common use-case would be glyph-caching for fonts.

\subsubsection{Tessellation}

A consistently used pre-computation model is tessellation, often called triangulation. This approach turns complex paths into discrete triangles for use in a traditional rendering engine. Tessellation facilitates easy integration with any GPU rendering engine and requires little in terms of GPU features.\\

\smallsvg
% Path
{assets/Lyon_Logo.svg}
% Caption
{Lyon}
% Attribution
{\href{https://github.com/nical/lyon}{Lyon, MIT/Apache 2.0}}

A promising library for vector tessellation is \href{https://github.com/nical/lyon}{\textit{Lyon}}, written in Rust. Lyon is potentially the best and has contributed novel algorithms to path tessellation\cite{Lyon}. You can also find tessellation integrated in Microsoft's Direct2D API\cite{D2D_Tess}.

\subsection{Modern technologies}

Presently, analytic rendering with GPU pipelining is fairly onerous, but promising. The solutions have been as divergent as they are experimental, but majorly lacking context.\\

\href{https://skia.org}{\textit{Skia}} is a popular hardware-accelerated library that is commercially backed by Google. Fortunately, the library is available open source. It is the currently used for rendering in the Mozilla Firefox and Google Chrome web browsers, among others. It is an established standard.\\

\smallsvg
% Path
{assets/Skia_Project_Logo.svg}
% Caption
{Skia}
% Attribution
{\href{https://skia.org}{https://skia.org/, Fair use}}

Other open source alternatives display confidence, but lack performance comparison. Those seeking to integrate these libraries have little more than suspicion or narrow benchmarking to encourage decision-making.\\

\href{https://github.com/servo/pathfinder}{\textit{Pathfinder 3}} is a modern, sophisticated 2D renderer designed for vector and font rendering. It is a tiling scheme fitted with a traditional rasterization pipeline. It was slighted to be used in the \href{https://servo.org/}{Servo} mission, which shares code with Mozilla Firefox, to be an embeddable web engine.\\

\smallsvg
% Path
{assets/Pathfinder_Logo.svg}
% Caption
{Pathfinder 3}
% Attribution
{\href{https://github.com/servo/pathfinder}{Pathfinder, MIT/Apache 2.0}}

Next, we have two experimental technologies which may be more arduous and require more ceremony to run and measure.\\

\href{https://github.com/linebender/piet-gpu}{\textit{piet-gpu}} is an experimental prototype 2D GPU renderer featuring a compute-centric pipeline. While its main motivation is for graphic user interfaces, it shares many of the (good) similarities with Pathfinder 3. The research has contributed impressive results, namely with \href{https://raphlinus.github.io/rust/graphics/gpu/2020/06/12/sort-middle.html}{leveraging a sort-middle GPU architecture}.\\

\href{https://fuchsia.googlesource.com/fuchsia/+/refs/heads/main/src/graphics/lib/compute/spinel/README.md}{\textit{Spinel}} is an enigmatic marvel being made by Google. The technology is likely to be integrated into Skia, and follows a similar role. For now, the project is experimental and locked behind Google's operating system: Fuchsia. Simply getting it to run or working fluidly with has been oppressively difficult. However, the \href{https://fuchsia.googlesource.com/fuchsia/+/refs/heads/main/src/graphics/lib/compute/spinel/README.md}{\emph{promises}} it makes are exciting.

\subsection{Benefits of a GPU analysis}

The benefits of a GPU analysis are two-fold. Firstly, to survey the current 2D GPU path-rendering capability by benching current technologies. This would provide optics on outlying behavior which can improve lacking performance. Secondly, a set of qualitative measures can accentuate concerns and set modern expectations.
