\appendix
\addcontentsline{toc}{section}{Appendix}

\section{File storage savings - SVG vs PNG}
\label{appendix:svg_vs_png}
This explains the methodology for Table \ref{tab:impossible_cubes_stats}.
All files used to replicate results can be found at \href{https://github.com/simbleau/simbleau/tree/research/2021-thesis-gpu-analysis}{https://github.com/simbleau/simbleau/tree/research/2021-thesis-gpu-analysis}. We used linux system binaries and \textit{inkscape} for SVG $\rightarrow$ PNG file exporting.\\

First we selected the file \textit{Impossible\_Cubes.svg} for analysis, but we could have used any SVG. Next, we parsed the SVG file header for viewport information to obtain the canonical size the SVG was saved in. \textit{Impossible\_Cubes.svg} contained the following metadata, indicating it is an 375 x 429 image originally.

\begin{verbatim}
viewBox="0 0 374.95 429.34"
\end{verbatim}

Thus, to export at $1x$ scale, we used the following \textit{inkscape} command:

\begin{verbatim}
inkscape -w 375 -h 429 Impossible_Cubes.svg -e Impossible_Cubes.png
\end{verbatim}

File savings were measured in bytes with the formula $f(x,y)=100(1-\frac{x}{y})$, where $f(x,y)$ is the percentage of storage savings, $x$ is the amount of original (SVG) file bytes, and $y$ is the new amount of (PNG) file bytes.