\section{Literature review}

Modern 2D GPU vector graphic rendering on the GPU is a culmination of impressive research. This includes GPU curve rendering\cite{Loop05}, random-access vector graphics\cite{Nehab08}, a massively parallel pipeline\cite{Ganacim14}, novel scan-line algorithms\cite{Li16}, triangle tessellation\cite{Silva18}, and GPU architecture leveraging\cite{Levien20}.

\subsection{Resolution Independent Curve Rendering Using Programmable Graphics Hardware\cite{Loop05}}

Presented at SIGGRAPH Asia in 2005 and published in the ACM Transactions on Graphics (TOG), this was the first major piece to show parallelism in resolution independent graphics. Prominent Microsoft researchers Charles Loop and James Blinn presented a method that leverages the parallelism of programmable graphics hardware to achieve high performance. The method constructed vector images from mosaics of triangulated Bézier control points, teasing the power of GPU parallelism.

\subsection{Random Access Vector Graphics\cite{Nehab08}}

Presented at SIGGRAPH Asia and published by the ACM ToG in 2008, Diego Nehab and Hugues Hoppe created the first tiling approach for vector graphics. This is a pre-computation model which enhanced the image model greatly to render static vector graphics (with support for transformations) at interactive rates. The pre-computation method was impressive but was not fast enough for dynamism. Their approach encoded \textit{Ghostscript Tiger} in 0.44 seconds, and \textit{Ghostscript Tiger} is not a challenging render by today's standards.\\

\smallsvg
% Path
{assets/Ghostscript_Tiger.svg}
% Caption
{Ghostscript Tiger}
% Attribution
{\href{http://www.gnu.org/licenses/agpl.html}{Ghostscript authors, AGPL}}

\subsection{High Performance Software Rasterization on GPUs\cite{Laine11}}
Authors Samuli Laine and Tero Karras, researchers from NVIDIA, had their work published by the ACM SIGGRAPH Symposium in 2011. Their implementation "CUDA Raster" was easily extensible and featured a completely software-based graphics pipeline on a GPU, which obeyed ordering constraints from traditional rendering pipelines. Their performance improved by a factor of 2–8x compared to a top of the line GPU in 2011. This research did not focus on vector graphics, but similarly set up a lot of theory behind compute-based parallel rendering.\\

\subsection{GPU-accelerated Path Rendering\cite{Kilgard12}}
Presented at SIGGRAPH Asia and published by the ACM ToG in 2012, Mark J. Kilgard and Jeff Bolz released one of the first analytic rendering approaches to 2D vector graphics on the GPU. Their approach builds upon existing techniques for curve rendering using the stencil buffer and explicitly decouples the stencil step to determine a path's filled or stroked coverage.\\

\subsection{Massively Parallel Vector Graphics\cite{Ganacim14}}
Published in the ACM ToG and Proceedinds of ACM SIGGRAPH Asia 2014, Ganacim et al. reach a breakthrough in vector graphics parallelization. It is noteworthy that Diego Nehab is involved with this research (renowned expert on vector graphics). This solution solves the dynamism of previous models and optimizes the pipeline majorly. The rendering pipeline is divided into two components: A preprocessing component builds a novel, the shortcut tree, and a rendering component which processes all samples and pixels in parallel. Tree construction is efficient and parallel at the segment level, enabling dynamic vector graphics.\\

\subsection{Efficient GPU Path Rendering Using Scanline Rasterization\cite{Li16}}
Published in the ACM ToG and presented in SIGGRAPH Asia 2016, Li et al. release a major milestone in vector graphics rendering. The solution is parallel, optimized, and supports dynamism. the method is parallelized over boundary fragments (pixels intersecting the path boundary) and non-boundary pixels are processed in bulk, similar to CPU scanline rasterizers. This novel scaline algorithm significantly saves on the the amount of winding number computations. To this day, it remains as one of the fastest methods for rasterization and GPU efficiency.

\subsection{Discrete Triangle Tessellation\cite{Silva18}}
Nicolas Silva exhibits at RustFest 2018 which highlights the benefits of Lyon and how it can assist in path tessellation for GPU rendering. The algorithms used in Lyon are performant and novel.

\subsection{Sort-Middle Architecture\cite{Levien20}}
Dr. Raph Levien publishes a blog piece on a sort-middle architecture advancement for 2d graphic rendering. Merged into the project \textit{piet-gpu}, the architecture explains the motivation for a compute-centric pipeline to maximizes parallelism by sorting in the middle of the pipeline. The performance claims and results listed describe a \emph{signifcant} breakthrough in the architecture, despite lacking the same level of prestigious publication as previous literatures listed here.