\section{Introduction}\label{sec:introduction}

\subsection{Problem Statement}\label{sec:problem_statement}
The plumbing of video-card architecture has been historically optimized for triangle arithmetic and data flow. This specificity has led to rigid render pipelines and difficulty with generalized parallel computation. Hence, this ingrained rigidity is why vector graphics are considered GPU-hostile. Given that vector images are formatted implicitly as ``equations'' rather than discrete pixel rows of color data, a different approach is necessary for GPU rendering; flexibility is required to parallelize the rasterization processing of vector graphics on the GPU. Moreover, processing implicit data adds a level of indirection, prompting a substantial overhead for rendering not similarly experienced in raster graphics.\medskip

With the rise of support for compute kernels and low-level GPU architecture access over the past few years, friction with general-purpose GPU computing is fading. With this new access to low-level hardware features comes experimentation. The field of hardware-accelerated vector graphics seems optimistic, with new attempts to leverage these features. However, there is a noticeable lack of comparison between techniques and libraries which gauge the modern rendering capability. This lack of comparison is partly due to the highly complex strategy required to precisely sample GPU metrics. Therefore, relative comparisons, time metrics, and \textit{Big-O} is typically provided as a decent proxy.\medskip

Analyzing the performance of vector graphics on the GPU is \emph{hard}. Various renderers and approaches are tuned for fonts, mobile power consumption, or other scarce computer resources. Given new technologies attempting to solve these issues, it is an appropriate step to respond with an analysis of the model and how to measure it. We can provide optics, encourage further research, and de-obfuscate the field by providing an analytic framework to measure hardware-accelerated vector graphics.

\subsection{Research Outline}\label{sec:research_outline}
This research thesis will begin by examining prior approaches to vector rendering in a literature review found in section \cref{sec:literature_review}, with attempts to qualify significance where possible. Afterward, we consolidate theories we reserve on vector graphics with synthesized theories. These theories accentuate considerations for evaluating vector rendering efficacy through functional and non-function requirement, as well as constitute the basis for the methodology and architecture of our framework. We supply diagrams and procedures which justify the design goals of our architecture. Finally, we provide results to prove our product through trial in a test case.\medskip

Ultimately, our product is theory and a GPU analysis framework that orchestrates sequential execution of small, independent test containers, augmented with atomically synchronized monitors to collect measurements in partial satisfaction of our requirements. Furthermore, our product is an extensible, open-source benchmarking framework, befitting the rapidly changing field of hardware-accelerated vector graphics.