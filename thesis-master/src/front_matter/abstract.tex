\vspace*{0.5in}
\begin{center}
    \textbf{Abstract}\addcontentsline{toc}{section}{Abstract}\\[2ex]
    \bigskip
    \begin{spacing}{1.1}
    \begin{adjustwidth}{3cm}{3cm}
    \begin{center}
    \thesistitle
    \end{center}
    \end{adjustwidth}
    \end{spacing}
    \normalsize
    \bigskip
    Spencer C. Imbleau\\
    B.S., Western Carolina University\\
    M.S., Appalachian State University\\
    \bigskip
    Chairperson: R. Mitchell Parry, Ph.D
\end{center}

\begin{doublespace}
With the rising support of compute kernels and low-level GPU architecture access over the past few years, friction with general-purpose GPU computing is fading. With new accessibility, new analytics methods for hardware-accelerated vector rasterization are being tried with new leverage. There are compelling reasons to optimize performance given the resolution-independent imaging model and inherent benefits. However, there is a noticeable lack of comparison between algorithms, techniques, and libraries which gauge the modern rendering capability. Analyzing the performance of vector graphics on the GPU is challenging, primarily when various technologies may compete for differing scarce computer resources. This thesis examines the contention found with modern vector graphic rendering and expands on analysis techniques used to deobfuscate efficacy by providing an analytic benchmarking framework for hardware-accelerated renderers.
\end{doublespace}